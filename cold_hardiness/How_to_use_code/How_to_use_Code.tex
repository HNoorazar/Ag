\documentclass[12pt]{article}
\pdfoutput=24
\usepackage{amsmath}
\usepackage{amsthm}
\usepackage{esvect}

\usepackage[toc,page]{appendix}

\usepackage{leftidx}
\usepackage{color}
\usepackage{framed, color}
\usepackage{multirow}
% \usepackage{pdfpages}
\usepackage{multicol}
\usepackage{wrapfig, booktabs}
\usepackage[font=small,skip=0pt]{caption}

\usepackage[utf8]{inputenc}
\usepackage{mathtools, hyperref}
\usepackage{hyperref}
\hypersetup{
    colorlinks=true,
    linkcolor=cyan,
    filecolor=cyan,
    urlcolor=blue,
    citecolor=red,
}

\usepackage{cleveref}
\usepackage{commath}
\usepackage{enumitem}
\usepackage{amssymb}
\renewcommand{\qedsymbol}{$\blacksquare$}
%%%%%%%%%%%
%%%%%%%%%%%  User Defined Commands. (macros)
%%%%%%%%%%%

\definecolor{mgreen}{RGB}{25,147,100}
\definecolor{shadecolor}{rgb}{1,.8,.1}
\definecolor{shadecolor2}{RGB}{245,237,0}
\definecolor{orange}{RGB}{255,137,20}
\definecolor{orange}{RGB}{245,37,100}

%%%%%%%%%%%%%%%%%
%%%%%%%%%%%%%%%%%%%%%%%        Theorem Styles
%%%%%%%%%%%%%%%%%
\theoremstyle{plain}
\newtheorem{theorem}{Theorem}[section]
\newtheorem{prop}{Proposition}[section]
\newtheorem{corr}{Corollary}[section]
\theoremstyle{definition}
\newtheorem{definition}{Definition}[section]
\newtheorem{lemma}[theorem]{Lemma}
\theoremstyle{definition}
\newtheorem{remark}{Remark}[section]
\newtheorem{fact}{Fact}[section]

\usepackage[english]{babel}
\usepackage{babel,blindtext}
\newtheorem{corollary}{Corollary}[theorem]
\newtheorem{exmp}{Example}[section]
\usepackage{fullpage}
\usepackage{amsfonts}
\usepackage{lscape}
\usepackage{bbm}

\usepackage[x11names, dvipsnames]{xcolor}

\usepackage{todonotes} % makes problem with xcolor package, if loaded before
\usepackage{cite}
\usepackage{verbatim}
\usepackage{bm}

\DeclareMathOperator*{\argmax}{arg\,max}
\usepackage[ textwidth = 15cm]{geometry}
\providecommand{\keywords}[1]{\textbf{\textit{Keywords:---}} #1}

\usepackage[T1]{fontenc}
\usepackage[utf8]{inputenc}
\usepackage{authblk}
\usepackage{cite}


%%%%%%%%%%%%%%%%%%%%%%%%%%%%%%%%%%%%%%%%%%%%%%%%%%%%%
%%%%%%%%%%%%%%%%%
%%%%%%%%%%%%%%%%% Code style
%%%%%%%%%%%%%%%%%
\usepackage{lipsum}
\newlength{\seplinewidth}
\newlength{\seplinesep}
\setlength{\seplinewidth}{1mm}
\setlength{\seplinesep}{2mm}


\colorlet{sepline}{PaleVioletRed3}
\newcommand*{\sepline}{%
  \par
  \vspace{\dimexpr\seplinesep+.5\parskip}%
  \cleaders\vbox{%
    \begingroup % because of color
      \color{sepline}%
      \hrule width\linewidth height\seplinewidth
    \endgroup
  }\vskip\seplinewidth
  \vspace{\dimexpr\seplinesep-.5\parskip}%
}

\usepackage[english]{babel}
\usepackage[utf8]{inputenc}
\usepackage{fancyhdr}
\pagestyle{fancy}
\fancyhf{}
\rhead{}
\lfoot{Page \thepage}
\rfoot{h.noorazar@yahoo.com}
\renewcommand{\footrulewidth}{2pt}
\futurelet\TMPfootrule\def\footrule{{\color{Cerulean}\TMPfootrule}}

%\lhead{Help}
\renewcommand{\headrulewidth}{0pt}
%\futurelet\TMPfootrule\def\headrule{{\color{Cerulean}\TMPfootrule}}
\def\code#1{\textbf{\texttt{#1}}}
\def\vari#1{{\color{Cerulean}{\textbf{\texttt{#1}}}}}
\def\func#1{{\color{Maroon}{\textbf{\texttt{#1}}}}}
\newenvironment{coded}{\color{blue}\code}


%%%%%%%%%%%%%%%%%%%%%%%%%%%%%%%%%%%%%%%%%%%%%%%%%%%%

\newcommand*{\affaddr}[1]{#1} % No op here. Customize it for different styles.
\newcommand*{\affmark}[1][*]{\textsuperscript{#1}}
%\newcommand*{\email}[1]{\texttt{#1}}

\title{\textbf{How to use code}}
\usepackage{authblk}
% \author[1]{Kirti Rajagopalan \thanks{kirtir@wsu.edu}}
% \affil[1]{CAHNR, Washington State University}
\author{}
\date{}
\providecommand{\keywords}[1]{\textbf{\textit{Keywords:}} #1}

\begin{document} 
\maketitle
\begin{abstract}
This model is distributed by the Washington 
State University Viticulture  Program, 
located at WSU-IAREC, 24106 North Bunn Rd., Prosser, WA 99350, USA.  
This model is also currently available at WSU's AgWeatherNet 
(weather.wsu.edu).  A login account is required to view the model, 
but it is currently free of charge.   More information on cold 
hardiness modeling and monitoring can be found at: 
\href{http://wine.wsu.edu/research-extension/weather/cold-hardiness}{http://wine.wsu.edu/research-extension/weather/cold-hardiness.}

\end{abstract}

\keywords{}

\section{How to use code part of the excel file}

The WSU cold hardiness model (version 2) developed by 
John Ferguson is written in Visual Basic as a macro. 
Press Ctl-m to run the model, then press Ctl-g to 
graph the output. The model will read the variety-specific 
constants from the \vari{input\_parameters} worksheet. 
Weather data will be read from the \vari{input\_temps}
 worksheet. All temperatures are in degrees Celsius. 
Results will be placed in the \vari{model\_output} worksheet. 
Running the model will clear any previous results that are
 in the \vari{model\_output} worksheet. 

If you get an error about running macros and security, 
you will need to change that setting.

Depending on your version of Excel, you  may need to open 
Excel without the model, change the \textbf{Tools > Macro > Security} 
to low, click ok, then open the model. Newer 
versions  of Excel 
will  give a security warning that macros have been 
disabled. Select  \textbf{Options > Enable}  this content.

The model code may be viewed and/or edited by 
clicking on \textbf{Tools  >  Macro  >  Macros >  model  > Edit. Newer 
versions, select View > Macros > View macros > Edit}.

You should modify the pertinent columns of the \vari{input\_temps}
 worksheet to suit your local conditions. The current default data is 
 from Prosser, Washington. Change the data in the \vari{input\_parameters}
 worksheet to the variety of interest by copying the pertinent line 
 from \vari{variety\_parameters}. If you have any measured cold 
 hardiness (Hc) values, put them in column H of 
 the \vari{input\_temps} worksheet and change that 
 column heading to a correct description of your data. 
 Keep all yellow column headings intact. The default 
 observed Hc values are for Cabernet Sauvignon in Prosser.

Feel free to play with some of the  values  placed 
in the \vari{input\_parameters}  sheet to explore how 
the model works , get it to fit your observations,
 or model a variety not listed. If  the model shows 
 that deacclimation is delayed  at your location 
 it probably means the Ecodormancy\_boundary is too large.

The model uses integer day of year (jday) as the 
time variable, typically starting on day 250 
(September 7th) and counting past the first of the year to day 500 (May 15th).



\end{document}
