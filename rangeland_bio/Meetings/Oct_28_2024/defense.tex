\documentclass[serif, xcolor={dvipsnames}]{beamer} % serif, mathserif
\graphicspath{ {/Users/hn/Documents/01_research_data/RangeLand_bio/plots/} }
\usepackage[scaled=.9]{helvet} % platino activation
\usepackage{tabularx,multirow}
\usepackage{colortbl}
\RequirePackage{booktabs}
\colorlet{shadecolor}{gray!40}
\usepackage{chngcntr}
\usepackage{tcolorbox}
\setbeamertemplate{footline}[text line]{}
\setbeamertemplate{navigation symbols}{}
\usepackage{subcaption}
\usepackage{amsmath}
\usepackage{amsthm}
\usepackage{amsfonts}
\usepackage{amssymb}
\usepackage{calrsfs}
\usepackage{multicol}
%\captionsetup{font={small,stretch=0.80}}
\DeclareCaptionLabelFormat{andtable}{#1~#2  \&  \tablename~\thetable}
\usetheme{Berlin}
%\usetheme{Boadilla}
%\usetheme{Copenhagen}
%\usetheme{Ilmenau}
\usepackage{textpos}
\usepackage{etoolbox}
% \usepackage{mathptmx}
\usepackage{graphicx}
\usecolortheme{beaver}
\definecolor{mygreen}{cmyk}{0.82,0.11,1,0.25}
\definecolor{aliceblue}{rgb}{0.94, 0.97, 1.0}
\definecolor{azureWeb}{rgb}{0.94, 1.0, 1.0}
\definecolor{codegreen}{rgb}{0,0.6,0}
\definecolor{codegray}{rgb}{0.5,0.5,0.5}
\definecolor{codepurple}{rgb}{0.58,0,0.82}
\definecolor{backcolour}{rgb}{0.95,0.95,0.92}
\definecolor{aliceblue}{rgb}{0.94, 0.97, 1.0}
\definecolor{bittersweet}{rgb}{1.0, 0.44, 0.37}
\definecolor{Gray}{gray}{0.85}
% \definecolor{LCyan}{rgb}{0.88,1,1}
\definecolor{mgreen}{rgb}{0.0,0.5,0.0}
%% \definecolor{mgreen}{rgb}{0.2, 0.8, 1}
\definecolor{brickred}{rgb}{0.8, 0.25, 0.33}
\definecolor{bleudefrance}{rgb}{0.19, 0.55, 0.91}
\definecolor{AB}{rgb}{0.94, 0.97, 1.0}
\definecolor{AB}{rgb}{0.9,.81,.68}
\definecolor{azureWeb}{rgb}{0.94, 1.0, 1.0}
\definecolor{beaublue}{rgb}{0.74, 0.83, 0.9}
\definecolor{linen}{rgb}{0.98, 0.94, 0.9}
\definecolor{magnolia}{rgb}{0.97, 0.96, 1.0}
\definecolor{moccasin}{rgb}{0.98, 0.92, 0.84}
\definecolor{navajowhite}{rgb}{1.0, 0.87, 0.68}
\definecolor{palecornflowerblue}{rgb}{0.67, 0.8, 0.94}
\setbeamertemplate{blocks}[rounded][shadow=false]
\addtobeamertemplate{block begin}{\pgfsetfillopacity{0.8}}{\pgfsetfillopacity{1}}
\setbeamercolor{structure}{fg=mygreen}
\setbeamercolor*{block title example}{fg=blue!50,
bg= blue!10}
\setbeamercolor*{block body example}{fg= blue,
bg= blue!5}
\usepackage{ulem}
\usepackage{cancel}
\usefonttheme{professionalfonts}
{\vspace*{.10mm}}
\usetheme[height=10mm]{Rochester}
\title{Rangeland Biology}
\author{Hossein Noorazar}
\institute[WSU]{Washington State University}
\date{\today}
\titlegraphic{\includegraphics[width=1cm]{cat}}
\begin{document}
\maketitle
%\addtobeamertemplate{frametitle}{}{%}
%%%%%%%%%%%%%%%%%%%%
%%%%%%%%%%%%%%%%%%%%
%%%%%%%%%%%%%%%%%%%%
\iffalse
\begin{frame}
\frametitle{Curious}
\begin{minipage}{.7\textwidth}
\centering
\includegraphics[width=.9\linewidth]{area_TS}~%
\end{minipage}%
\hspace{-0.5cm}
\begin{minipage}{.3\textwidth}
\resizebox{\columnwidth}{!}{
\begin{tabular}{ll}
\bottomrule
\rowcolor{shadecolor}\multicolumn{2}{c}{\bf 2012} \\
\hline
{\bf Vegetation} & {\bf Area (Km$^2$)} \\
\bottomrule
\rowcolor{blue}{\small \textcolor{white}{Barren-Rock}} & \textcolor{white}{1,133} \\
\rowcolor{Green} \textcolor{white}{Conifer} & \textcolor{white}{118} \\
\rowcolor{red} \textcolor{white}{Grassland} & \textcolor{white}{165,181}\\
\rowcolor{Cyan}\textcolor{white}{Hardwood} & \textcolor{white}{137}\\
\rowcolor{magenta}\textcolor{white}{Riparian} & \textcolor{white}{128}\\
\rowcolor{yellow}\textcolor{black}{Shrubland} & \textcolor{black}{82,872}\\
\rowcolor{black}\textcolor{white}{Sparse} & \textcolor{white}{64}\\
\toprule
\end{tabular}
}
\end{minipage}
\end{frame}
%%%%%%%%%%%%%%%%%%%%
%%%%%%%%%%%%%%%%%%%%
%%%%%%%%%%%%%%%%%%%%
\begin{frame}
 \frametitle{Curious}
\includegraphics[width=1\linewidth]{area_TS_allVegs}~%
\end{frame}
%%%%%%%%%%%%%%%%%%%%
%%%%%%%%%%%%%%%%%%%%
%%%%%%%%%%%%%%%%%%%%
\begin{frame}
 \frametitle{Curious}
{\hspace*{-5mm}}
\begin{minipage}{.5\textwidth}
\includegraphics[width=1\linewidth]{area_TS_no_2012}~%
\end{minipage}%
\hspace{.21cm}
\begin{minipage}{.5\textwidth}
\includegraphics[width=1\linewidth]{area_TS_no_1984_2012}~%
\end{minipage}
\end{frame}
\fi
%%%%%%%%%%%%%%%%%%%%
%%%%%%%%%%%%%%%%%%%%
%%%%%%%%%%%%%%%%%%%%
\begin{frame}
\frametitle{Trend Statistics}
% \begin{itemize}[<+->]
% \end{itemize}
{\bf Using MK test (Sen's slope)}
\begin{table}[!ht]
\centering
\captionsetup{singlelinecheck=false, format=hang}
\label{tab:Trendcounts}
% \vspace{-.15in}
\begin{tabular}{lllll}
\bottomrule
\rowcolor{shadecolor} \textbf{Trend} & 
 greening & browning & no trend \\ 
\textbf{original MK}  & 21,014 & 24 & 6,188 \\
\rowcolor{shadecolor} \textbf{MK Rao}  & 20,231 & 24 & 6,971 \\
\textbf{MK Yue}  &  25,115 & 91 & 2,020 \\
\rowcolor{aliceblue}\textbf{Spearman}  & 21,703 & 27 & 5,496\\
\toprule
\end{tabular}
\end{table}
\begin{tcolorbox}
{\scriptsize{\bf Note.} Spearman does not provide a label for trend. The numbers
in this table based on whether $p$-value$<$0.05 or not and whether $\rho$'s are positive or negative.}
\end{tcolorbox}
\end{frame}
%%%%%%%%%%%%%%%%%%%%%
%%%%%%%%%%%%%%%%%%%%%
%%%%%%%%%%%%%%%%%%%%%
\begin{frame}
\begin{center}
\includegraphics[height=1\textheight]{noTrend_green_locs}
\end{center}
\end{frame}
%%%%%%%%%%%%%%%%%%%%%
%%%%%%%%%%%%%%%%%%%%%
%%%%%%%%%%%%%%%%%%%%%
\begin{frame}
 \frametitle{Three examples (two extremes)}
\begin{center}
\includegraphics[height=0.8\textheight]{three_trends}
\end{center}
\end{frame}
%%%%%%%%%%%%%%%%%%%%%
%%%%%%%%%%%%%%%%%%%%%
%%%%%%%%%%%%%%%%%%%%%
\begin{frame}
 \frametitle{Sen's Slope and Outliers}
\begin{center}
\includegraphics[height=0.6\textheight]{robustSensSlope}
\end{center}
\end{frame}
%%%%%%%%%%%%%%%%%%%%%
%%%%%%%%%%%%%%%%%%%%%
%%%%%%%%%%%%%%%%%%%%%
\begin{frame}
\frametitle{Trends by Sen's Slope}
\begin{center}
\includegraphics[height=1\textheight]{sensSlopes_centerColorBar}
\end{center}
\end{frame}
%%%%%%%%%%%%%%%%%%%%
%%%%%%%%%%%%%%%%%%%%
%%%%%%%%%%%%%%%%%%%%
\begin{frame}
\frametitle{Greening FIDs by Yue, missed by original MK-test}
\begin{center}
\includegraphics[height=.9\textheight]{yue/greenYue_missedOriginal}
\end{center}
\end{frame}
%%%%%%%%%%%%%%%%%%%%
%%%%%%%%%%%%%%%%%%%%
%%%%%%%%%%%%%%%%%%%%
\begin{frame}
\frametitle{Spearman and Kendall's}
Trends based on Spearman's rank and Kendall's $\tau$
\centering
\includegraphics[height=0.75\textheight]{Spearman_tau}
\end{frame}
%%%%%%%%%%%%%%%%%%%%%%%%%%%%%
\begin{frame}
\frametitle{Sen's Slope and Median Differences Slope}
Sen's Slope and Median Differences Slope.
\begin{center}
\includegraphics[height=0.75\textheight]{Sens_MedianDiffSlope}
\end{center}
\end{frame}
%%%%%%%%%%%%%%%%%%%%
%%%%%%%%%%%%%%%%%%%%
%%%%%%%%%%%%%%%%%%%%
\begin{frame}
\frametitle{Median ANPP Percentage Change}
Median ANPP percentage change
\begin{center}
\includegraphics[height=0.9\textheight]{medianNPP_percChange}
\end{center}
\end{frame}
%%%%%%%%%%%%%%%%%%%%
%%%%%%%%%%%%%%%%%%%%
%%%%%%%%%%%%%%%%%%%%
\begin{frame}
\frametitle{Trends Distribution}
\begin{center}
\includegraphics[height=1\textheight]{trend_distributions}
\end{center}
\end{frame}
%%%%%%%%%%%%%%%%%%%%
%%%%%%%%%%%%%%%%%%%%
%%%%%%%%%%%%%%%%%%%%
\begin{frame}
\frametitle{Regression - preamble}
\begin{itemize}[<+->]
\item We do regression on Greening locations
\item Temporal coverage
\begin{table}[!ht]
\centering
\captionsetup{singlelinecheck=false, format=hang}
\label{tab:NPPWeatherYears}
% \caption{The years for which we have data.}
\begin{tabular}{lll}
\bottomrule
\rowcolor{shadecolor} \textbf{ANPP years} &  1984 - 2023   \\ 
\textbf{Weather years}  & 1979 - 2022 \\
\toprule
\end{tabular}
\end{table}
\end{itemize}
\end{frame}
\begin{frame}
\frametitle{Feature Importance - Monthly variables}
\begin{itemize}
\item Metrics
\begin{table}[!ht]
\hspace{-.5in}
\captionsetup{singlelinecheck=false, format=hang}
\label{tab:Metrics}
% \caption{NPP = $f(T, P)$}
\begin{tabular}{lll}
\bottomrule
\cellcolor{palecornflowerblue}{\small f(T, P)} & \small{ $\mathbf{R^2}$}  & \small{ $\mathbf{RMSE}$} \\ 
\rowcolor{shadecolor} 
{\bf train} & 0.6746  &  543\\ 
{\bf test}  & 0.6744  & 543\\
\toprule
\end{tabular}\quad
%%%%%%%%%%%%%%%%%%%%%
\begin{tabular}{lll}
\bottomrule
\cellcolor{palecornflowerblue}{\small f(T, P, RH)} & $\mathbf{R^2}$  & \small{$\mathbf{RMSE}$} \\ 
\rowcolor{shadecolor} 
{\bf train} &  0.707  &  515\\ 
{\bf test}  & 0.706  & 516\\
\toprule
\end{tabular}
\end{table}
\item Coefficients of variables for Grasslands 
\begin{table}[!ht]
\captionsetup{singlelinecheck=false, format=hang}
\label{tab:regressionCoeffs}
\hspace{-.53in}
\begin{tabular}{lllllllllllll}
\bottomrule
\rowcolor{shadecolor} &  1 & 2 & 3 & 4 & 5 & 6 & 7 & 8 & 9 & 10 & 11 & 12  \\ 
{\bf T} & \tiny{-174} &  \tiny{-157} &  \tiny{-213} &  \tiny{-136} &  \tiny{83} &  \tiny{201} &  \tiny{-103} &  \tiny{55} &  \tiny{104} &  \tiny{54} &  \tiny{119} & \tiny{-73} \\
{\bf P} &   \tiny{125} &  \tiny{98} &  \tiny{105} &  \tiny{144} &  \tiny{204} &  \tiny{181} &  \tiny{110} &  \tiny{131} &  \tiny{86} &  \tiny{40} & \tiny{ 69} &  \tiny{141} \\
\toprule
\end{tabular}\\
%%%%%%%%%%%%%%%%%%%%%
\hspace*{-.5in}
\begin{tabular}{lllllllllllll}
\bottomrule
\rowcolor{shadecolor} &  1 & 2 & 3 & 4 & 5 & 6 & 7 & 8 & 9 & 10 & 11 & 12  \\ 
{\bf T}    & \tiny{-61} & \tiny{-142} & \tiny{-137} & \tiny{-214} & \tiny{-67} & \tiny{324} & \tiny{6} & \tiny{-30} & \tiny{62} & \tiny{60} & \tiny{156} & \tiny{-11}\\
{\bf P}    & \tiny{74} & \tiny{72} & \tiny{87} & \tiny{116} & \tiny{162} & \tiny{121} & \tiny{82} & \tiny{134} & \tiny{72} & \tiny{53} & \tiny{18} & \tiny{90} \\ 
{\bf RH} &  \tiny{5} & \tiny{3} & \tiny{72} & \tiny{-8} & \tiny{-50} & \tiny{169} & \tiny{148} & \tiny{0} & \tiny{44} & \tiny{-90} & \tiny{106}  & \tiny{138} \\ 
\toprule
\end{tabular}
\end{table}
\end{itemize}
\end{frame}
\begin{frame}
\frametitle{Feature Importance - XGBoost and SHAP}
\hspace{.2in}
\includegraphics[height=.75\textheight]{XGBoost_feature_importance}
\includegraphics[height=.75\textheight]{SHAP_XGBoost_train}
\begin{table}[!ht]
\label{tab:XGBErrMetrics}
\begin{tabular}{lll}
\bottomrule
\cellcolor{palecornflowerblue}{\tiny f(T, P)} &  \tiny{$\mathbf{R^2}$} & \tiny{$\mathbf{RMSE}$}  \\ 
\rowcolor{shadecolor} 
\tiny{\bf train}   &  \tiny{0.94}  &  \tiny{239}\\ 
 \tiny{\bf test}  & \tiny{0.91}  & \tiny{284}\\
\toprule
\end{tabular}\qquad
\end{table}
\end{frame}
\end{document}