\documentclass[serif, xcolor={dvipsnames}]{beamer} % serif, mathserif
\usepackage[natbib=true,style=authoryear,backend=bibtex,useprefix=true,giveninits=true]{biblatex}
\DeclareNameAlias{author}{last-first}
\addbibresource{./beamer_Refs.bib}
\usepackage[capitalise]{cleveref}

\def\code#1{{\scriptsize\texttt{#1}}}
% hypersetup only changes the year in citation.
\hypersetup{
    colorlinks=true,
    linkcolor=white, % color1 : will be black
    filecolor=red,
    urlcolor=ForestGreen,
    citecolor=cyan,
    bookmarksopen=false,
    pdftitle={Title},
    pdfauthor={Author},
}

\DeclareCiteCommand{\cite}
  {\color{cyan}\usebibmacro{prenote}}%
  {\usebibmacro{citeindex}%
   \usebibmacro{cite}}
  {\multicitedelim}
  {\usebibmacro{postnote}
  }



\graphicspath{ {/Users/hn/Documents/01_research_data/RangeLand_bio/plots/yue} }
\usepackage[scaled=.9]{helvet} % platino activation
\usepackage{tabularx,multirow}
\usepackage{colortbl}
\RequirePackage{booktabs}

\colorlet{shadecolor}{gray!40}

\usepackage{chngcntr}
\usepackage{tcolorbox}

\setbeamertemplate{footline}[text line]{}
\setbeamertemplate{navigation symbols}{}
\usepackage{subcaption}
\usepackage{amsmath}
\usepackage{amsthm}
\usepackage{amsfonts}
\usepackage{amssymb}
\usepackage{calrsfs}
\usepackage{multicol}
%\captionsetup{font={small,stretch=0.80}}
\DeclareCaptionLabelFormat{andtable}{#1~#2  \&  \tablename~\thetable}


\usetheme{Berlin}
%\usetheme{Boadilla}
%\usetheme{Copenhagen}
%\usetheme{Ilmenau}


\usepackage{textpos}


\usepackage{etoolbox}
% \usepackage{mathptmx}
\usepackage{graphicx}
\usecolortheme{beaver}
\definecolor{mygreen}{cmyk}{0.82,0.11,1,0.25}
\definecolor{aliceblue}{rgb}{0.94, 0.97, 1.0}

\definecolor{codegreen}{rgb}{0,0.6,0}
\definecolor{codegray}{rgb}{0.5,0.5,0.5}
\definecolor{codepurple}{rgb}{0.58,0,0.82}
\definecolor{backcolour}{rgb}{0.95,0.95,0.92}

\definecolor{bittersweet}{rgb}{1.0, 0.44, 0.37}
\definecolor{Gray}{gray}{0.85}
% \definecolor{LCyan}{rgb}{0.88,1,1}
\definecolor{mgreen}{rgb}{0.0,0.5,0.0}
%% \definecolor{mgreen}{rgb}{0.2, 0.8, 1}
\definecolor{brickred}{rgb}{0.8, 0.25, 0.33}
\definecolor{bleudefrance}{rgb}{0.19, 0.55, 0.91}
\definecolor{AB}{rgb}{0.94, 0.97, 1.0}
\definecolor{AB}{rgb}{0.9,.81,.68}
\definecolor{azureWeb}{rgb}{0.94, 1.0, 1.0}
\definecolor{beaublue}{rgb}{0.74, 0.83, 0.9}
\definecolor{linen}{rgb}{0.98, 0.94, 0.9}
\definecolor{magnolia}{rgb}{0.97, 0.96, 1.0}
\definecolor{moccasin}{rgb}{0.98, 0.92, 0.84}
\definecolor{navajowhite}{rgb}{1.0, 0.87, 0.68}
\definecolor{palecornflowerblue}{rgb}{0.67, 0.8, 0.94}


\setbeamertemplate{blocks}[rounded][shadow=false]
\addtobeamertemplate{block begin}{\pgfsetfillopacity{0.8}}{\pgfsetfillopacity{1}}
\setbeamercolor{structure}{fg=mygreen}
\setbeamercolor*{block title example}{fg=blue!50,
bg= blue!10}
\setbeamercolor*{block body example}{fg= blue,
bg= blue!5}

\usepackage{ulem}
\usepackage{cancel}
\usefonttheme{professionalfonts}
{\vspace*{.10mm}}
\usetheme[height=10mm]{Rochester}


\title{Making a Case}
% \author{Making a Case}
% \institute[WSU]{Washington State University}
\date{\today}

\titlegraphic{\includegraphics[width=.8cm]{cat1}}

\begin{document}
\maketitle

%\addtobeamertemplate{frametitle}{}{%}
%%%%%%%%%%%%%%%%%%%%
%%%%%%%%%%%%%%%%%%%%
%%%%%%%%%%%%%%%%%%%%

\iffalse
\begin{frame}
\frametitle{Curious}
\begin{minipage}{.7\textwidth}
\centering
\includegraphics[width=.9\linewidth]{area_TS}~%
\end{minipage}%
\hspace{-0.5cm}
\begin{minipage}{.3\textwidth}
\resizebox{\columnwidth}{!}{
\begin{tabular}{ll}
\bottomrule
\rowcolor{shadecolor}\multicolumn{2}{c}{\bf 2012} \\
\hline
{\bf Vegetation} & {\bf Area (Km$^2$)} \\
\bottomrule
\rowcolor{blue}{\small \textcolor{white}{Barren-Rock}} & \textcolor{white}{1,133} \\
\rowcolor{Green} \textcolor{white}{Conifer} & \textcolor{white}{118} \\
\rowcolor{red} \textcolor{white}{Grassland} & \textcolor{white}{165,181}\\
\rowcolor{Cyan}\textcolor{white}{Hardwood} & \textcolor{white}{137}\\
\rowcolor{magenta}\textcolor{white}{Riparian} & \textcolor{white}{128}\\
\rowcolor{yellow}\textcolor{black}{Shrubland} & \textcolor{black}{82,872}\\
\rowcolor{black}\textcolor{white}{Sparse} & \textcolor{white}{64}\\
\toprule
\end{tabular}
}
\end{minipage}
\end{frame}

%%%%%%%%%%%%%%%%%%%%
%%%%%%%%%%%%%%%%%%%%
%%%%%%%%%%%%%%%%%%%%

\begin{frame}
 \frametitle{Curious}
\includegraphics[width=1\linewidth]{area_TS_allVegs}~%
\end{frame}
%%%%%%%%%%%%%%%%%%%%
%%%%%%%%%%%%%%%%%%%%
%%%%%%%%%%%%%%%%%%%%

\begin{frame}
 \frametitle{Curious}
{\hspace*{-5mm}}
\begin{minipage}{.5\textwidth}
\includegraphics[width=1\linewidth]{area_TS_no_2012}~%
\end{minipage}%
\hspace{.21cm}
\begin{minipage}{.5\textwidth}
\includegraphics[width=1\linewidth]{area_TS_no_1984_2012}~%
\end{minipage}
\end{frame}
\fi

%%%%%%%%%%%%%%%%%%%%
%%%%%%%%%%%%%%%%%%%%
%%%%%%%%%%%%%%%%%%%%

\begin{frame}
\frametitle{Tests and Their Powers}

\begin{itemize}% [<+->]
\item {\bf Null Hypothesis}
$H_0$: There is no trend

\item MK {\color{red}\emph{Original}} test: Accepts $H_0$ falsely due to auto-correlation

\item MK {\color{red}\emph{Adjusted Yue}}: Considers auto-correlation and rejects (too many) of the $H_0$. 
As a result statistical test power decreases
\end{itemize}
\end{frame}

%%%%%%%%%%%%%%%%%%%%%
%%%%%%%%%%%%%%%%%%%%%
%%%%%%%%%%%%%%%%%%%%%
\begin{frame}[t]
\frametitle{Yue Default Lag Most Ineffective}
Yue with (unknown) default lag should be avoided.

\begin{itemize}
\item We do not know what the default is (yet).
\item Greening locations w/ 99\% CL of Yue, \href{https://drive.google.com/drive/folders/1zdI_G_knRYy8PrTnLBIbmaWOvfWqZUxV?usp=sharing}{but dismissed by original MK test}

{\tiny 2,941 fields with slope less than 20. 111 fields with slope in 20-30. 12 more than 30.}


\bigskip
\begin{figure}
\includegraphics[height=.4\textheight]{/Users/hn/Documents/01_research_data/RangeLand_bio/plots/yue/Yue_99CL_TS_dismissedByOrig/slope_less20/FID_401_Yue99PercCL_dismissedOrig.pdf}
\caption{Lots of examples like this.}
\end{figure}

\end{itemize}
\end{frame}
%%%%%%%%%%%%%%%%%%%%%
%%%%%%%%%%%%%%%%%%%%%
%%%%%%%%%%%%%%%%%%%%%

\begin{frame}[t]
\frametitle{Trend Statistics}

% \begin{itemize}[<+->]
% \end{itemize}
{\bf Some MK test Statistics}

\begin{table}[!ht]
\centering
\captionsetup{singlelinecheck=false, format=hang}
\label{tab:Trendcounts}
% \vspace{-.15in}
\begin{tabular}{lllll}
\bottomrule
\rowcolor{cyan} 
{\bf trend} & {\bf greening} & {\bf browning} & {\bf no trend} \\ 
\rowcolor{shadecolor}
\textbf{original}  & 21,366 (17,017) & 24 (9) & 5,836 \\
\textbf{Yue Defau.}  &  25,187 (24,350) & 90 (56) & 1,949 \\
\rowcolor{shadecolor}
\textbf{Yue 0-lag}  & 21,366 (17,017) & 24 (9) & 5,836 \\
\textbf{Yue 1-lag}  &  20,064 (13,585) & 15 (6) &  7,147 \\
\textbf{Yue 2-lags}  & 19,077 (11,262) &  15 (7) & 8,134 \\
\textbf{Yue 3-lags}  & 19,289 (12,261) & 25 (10) & 7,912 \\
\toprule
\end{tabular}
\end{table}
\begin{tcolorbox}
{\scriptsize{\bf Note.} Numbers in parentheses are for 99\% confidence level.
}


{\scriptsize{\bf Note.} 0-lag and original are identical (not just their count.)
}

{\scriptsize{\bf \color{red}Q.} Any domain knowledge about lags?}

\end{tcolorbox}
\end{frame}

%%%%%%%%%%%%%%%%%%%%%
%%%%%%%%%%%%%%%%%%%%%
%%%%%%%%%%%%%%%%%%%%%

\begin{frame}
\frametitle{Some statistics}
Statistics about the locations labeled as greening by Yue (default lag) but dismissed by original MK-test.
\begin{table}[!ht]
\centering
\captionsetup{singlelinecheck=false, format=hang}
\label{tab:Trendcounts}
% \vspace{-.15in}
\begin{tabular}{lllll}
\bottomrule
\rowcolor{shadecolor} 
&  min & max & SE \\ 
\rowcolor{aliceblue} 
\textbf{Sen's slope} & 0.4 & 38 & 5.3 \\
\textbf{Kendal's $\tau$}  & 0.06 & 0.2 &  0.03 \\
\rowcolor{aliceblue} 
\textbf{Spearman}  & 0.06  & 0.4 & 0.05  \\
\textbf{\scriptsize $\Delta \text{ANPP}_{\text{medians}}$} & -49 & 1,736 & 257 \\
\rowcolor{aliceblue}\textbf{\scriptsize $\Delta \text{ANPP}_{\text{medians}}$ (\%)}  & -18.3 & 280 & 19.5 \\
\toprule
\end{tabular}
\end{table}
\end{frame}

%%%%%%%%%%%%%%%%%%%%%
%%%%%%%%%%%%%%%%%%%%%
%%%%%%%%%%%%%%%%%%%%%

%%%%%%%%%%%%%%%%%%%%%
%%%%%%%%%%%%%%%%%%%%%
%%%%%%%%%%%%%%%%%%%%%
\begin{frame}
\frametitle{Other Metrics Extremes}
\begin{center}
\includegraphics[height=0.7\textheight]{greenYue_extremeTau}
\end{center}
\end{frame}

%%%%%%%%%%%%%%%%%%%%%
%%%%%%%%%%%%%%%%%%%%%
%%%%%%%%%%%%%%%%%%%%%
\begin{frame}
\frametitle{Other Metrics Extremes}
\begin{center}
\includegraphics[height=0.7\textheight]{greenYue_extremesens_slope}
\end{center}
\end{frame}
%%%%%%%%%%%%%%%%%%%%%
%%%%%%%%%%%%%%%%%%%%%
%%%%%%%%%%%%%%%%%%%%%
\begin{frame}
\frametitle{Random Example}
\begin{center}
\includegraphics[height=0.7\textheight]{greenYue_ random}
\end{center}
\end{frame}
%%%%%%%%%%%%%%%%%%%%
%%%%%%%%%%%%%%%%%%%%
%%%%%%%%%%%%%%%%%%%%
\begin{frame}[t]
\frametitle{Yue 99\% Confidence Level - Extreme Sen's slopes}
\begin{center}
\includegraphics[height=.8\textheight]{Yue99PercCL_SensExtremes}
\end{center}
\end{frame}
%%%%%%%%%%%%%%%%%%%%
%%%%%%%%%%%%%%%%%%%%
%%%%%%%%%%%%%%%%%%%%
\begin{frame}[t]
\frametitle{Yue 99\% Confidence Level - Extreme Tau}
\begin{center}
\includegraphics[height=.8\textheight]{Yue99PercCL_TauExtremes}
\end{center}
\end{frame}
%%%%%%%%%%%%%%%%%%%%%
%%%%%%%%%%%%%%%%%%%%%
%%%%%%%%%%%%%%%%%%%%%

\begin{frame}[t]
\frametitle{Trade Off Ideas}
\vspace{-.15in}
What to do about the trade off? 
\begin{enumerate}
\item Increase confidence level from 95\% to higher values.

\item Use a 2nd method from literature to reject some of the fields labeled green by Yue. 
(It's normal to use 2 methods. I have seen a couple of examples)
or find the adjusted Spearman's. Normal spearman and original MK don't differ much \dots) 

\item Motivated by 2 examples above: slope$ > 3$? and $\tau> 0.15$? 

\item Use another modified MK-test to drop some of the Yue's greenings as suggested by~\cite{blain2013modified}.

\item Count \# of positive annual changes in NPP,
If more than $k$ positives, then declare greening? (not a good idea: )

\item Smooth the time-series and see how MK test/slope/$\tau$ is affected.
(Not good.)

\item We need a generic universal simple solution.
\end{enumerate}
\end{frame}

%%%%%%%%%%%%%%%%%%%%
%%%%%%%%%%%%%%%%%%%%
%%%%%%%%%%%%%%%%%%%%
\begin{frame}[t]
\vspace{-0.2in}
\frametitle{Yue Good Enough?}
{\small Maybe original MK is not too bad after all.
Examples with Sen's slope 30, 
rejected by original MK but Yue confidently (99\%) labels them as greening.{\footnote{\tiny I think~\cite{blain2013modified} mentioned autocorrelation is known in hydrology. Or was it Yue? Look at Blain and see how the variance adjustment is done. Learn about time series and autocorrelation and perhaps change their adjustment for more than 1 lag autocorrelation?
}}}
\begin{figure}[H] %htp
\centering
\captionsetup{singlelinecheck=false}
\subfloat{\includegraphics[width=.45\textwidth]{./Yue_99CL_TS_dismissedByOrig/slope_ge30/FID_25422_Yue99PercCL_dismissedOrig}
}\\
\subfloat{\includegraphics[width=.45\textwidth]{./Yue_99CL_TS_dismissedByOrig/slope_ge30/FID_25465_Yue99PercCL_dismissedOrig}
}\\
\subfloat{\includegraphics[width=.45\columnwidth]{./Yue_99CL_TS_dismissedByOrig/slope_ge30/FID_25467_Yue99PercCL_dismissedOrig}
}
\end{figure}
\end{frame}
%%%%%%%%%%%%%%%%%%%%
%%%%%%%%%%%%%%%%%%%%
\begin{frame}[t]
\frametitle{Autocorrelations: Extremes of DW statistics}
\vspace{-.2in}
\begin{figure}[H]
\centering
\includegraphics[width=.6\textwidth]{../NPP_TS_for_3FIDs_DWStatRange_sideBySide}
\captionsetup{singlelinecheck=false} 
\caption*{DW contradicting ACF test. Look into \color{red}{Ljung-Box Q test}.}
\label{fig:NPP_TS_for_3FIDs_DWStatRange_sideBySide}
\end{figure}
\end{frame}
%%%%%%%%%%%%%%%%%%%%%%%%%%%%%
%%%%%%%%%%%%%%%%%%%%%%%%%%%%%
%%%%%%%%%%%%%%%%%%%%%%%%%%%%%
\begin{frame}[t]
\frametitle{Autocorrelations: 3 random examples}
\vspace{-.2in}
\begin{figure}[H]
\centering
\includegraphics[width=.6\textwidth]{../NPP_TS_for_3FIDs_random_sideBySide}
\captionsetup{singlelinecheck=false} 
\caption*{Three random examples of DW vs ACF.}
\label{fig:NPP_TS_for_3FIDs_random_sideBySide}
\end{figure}
\end{frame}
%%%%%%%%%%%%%%%%%%%%%%%%%%%%%
%%%%%%%%%%%%%%%%%%%%%%%%%%%%%
%%%%%%%%%%%%%%%%%%%%%%%%%%%%%
\begin{frame}
\frametitle{Lags}

\begin{figure}[H]
\centering
\includegraphics[width=.8\textwidth]{../ACF_for_3random_FIDs_narow}
\captionsetup{singlelinecheck=false} 
\caption*{Three random examples: they show autocorrelation can go beyond 1 year. 
On a slightly, different tangent, there are other examples that showed 
no autocorrelation with this test 
while DW (another autocorrelation test) said there are strong autocorrelation.
Perhaps we need to test for autocorrelation in a given location,
and apply original MK if there is no autocorrelation? Or, just go with Sen's slope?}
\label{fig:ACF_for_3random_FIDs_narow}
\end{figure}
\end{frame}
%%%%%%%%%%%%%%%%%%%%%%%%%%%%%
%%%%%%%%%%%%%%%%%%%%%%%%%%%%%
%%%%%%%%%%%%%%%%%%%%%%%%%%%%%
\begin{frame}[t]
\frametitle{Google Drive}
\href{https://tinyurl.com/5n76ppzs}{Google drive all plots}

\begin{enumerate}
\item Yue's Greening locations with 99\% confidence level that are dismissed by original MK
are in subfolder \code{yue/Yue\_99CL\_TS\_dismissedByOrig} 

\item Original MKs (99\%CL) are in
subfolder \code{increasing\_originalMK\_slope\_TS}; \href{https://drive.google.com/drive/folders/197H1-FCRkS0gAqGr1DDreCU5n4NMWCwG?usp=share_link}{click here}

\item It seems originals with 99\% confidence level with Sen's slope greater than
30 have substantial trends in them. There are 2,260 fields in this set.

\item The number of fields whose Sen's slope is greater than 30 is 2,331.\\

\item There are 12 fields whose slopes are greater than 30 with no-trend label by MK test. (\code{original\_MK/steepSlope\_noTrend})


\end{enumerate}

\end{frame}
%%%%%%%%%%%%%%%%%%%%%%%%%%%%%
%%%%%%%%%%%%%%%%%%%%%%%%%%%%%
%%%%%%%%%%%%%%%%%%%%%%%%%%%%%
\begin{frame}
\frametitle{Other Metrics Extremes}
\begin{center}
\includegraphics[height=.7\textheight]{greenYue_extrememedian_ANPP_change_as_perc}
\end{center}
\vspace{-.2in}
{\scriptsize Maybe the difference between medians of NPP in the first and last decades is not the way to go about it?
But, it seems we need to toss some of the Yue's greening locations for sure.}
\end{frame}


%%%%%%%%%%%%%%%%%%%%
%%%%%%%%%%%%%%%%%%%%
%%%%%%%%%%%%%%%%%%%%

%%%%%%%%%%%%%%%%%%%%
\begin{frame}[t]
\frametitle{Trend Definition}
\begin{enumerate}
\item What is the point of killing autocorrelation?\\

\item Our visual inspection of plots do not detect trend, but MK test
sees a trend. Do our definitions agree? 

From Yue: Let 
\[X_t = {\color{red}\beta (t-1)} + {\color{mgreen} A_t},\]

where the red part is the linear trend part and the green part is the auto-correlation part; ${\color{mgreen}A_t = \mu_A+\rho (A_{t-1} - \mu_A) + \varepsilon_t},$ where $\varepsilon_t $ is noise, $\rho$ is correlation parameter, and
$\mu_A$ is the mean of autoregressive (AR) process!
\end{enumerate}
\end{frame}
%%%%%%%%%%%%%%%%%%%%%%%%%%%%%
%%%%%%%%%%%%%%%%%%%%%%%%%%%%%
%%%%%%%%%%%%%%%%%%%%%%%%%%%%%



%%%%%%%%%%%%%%%%%%%%
%%%%%%%%%%%%%%%%%%%%
%%%%%%%%%%%%%%%%%%%%


%%%%%%%%%%%%%%%%%%%%
%%%%%%%%%%%%%%%%%%%%
%%%%%%%%%%%%%%%%%%%%


%%%%%%%%%%%%%%%%%%%%
%%%%%%%%%%%%%%%%%%%%
%%%%%%%%%%%%%%%%%%%%
% \begin{frame}[allowframebreaks]
%        \frametitle{References}
%        \bibliographystyle{amsalpha}
%         \bibliography{./beamer_Refs.bib}
% \end{frame}
%%%%%%%%%%%%%%%%%%%%%
\begin{frame}[allowframebreaks, noframenumbering,t]
\frametitle{Bibliography}
\printbibliography
\end{frame}

\end{document}